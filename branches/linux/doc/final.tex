%======================================================================
%----------------------------------------------------------------------
%sergiobuj's ieee format template
%======================================================================
\documentclass[%
	%draft,
	%submission,
	%compressed,
	final,
	%
	%technote,
	%internal,
	%submitted,
	%inpress,
	%reprint,
	%
	%titlepage,
	notitlepage,
	%anonymous,
	narroweqnarray,
	inline,
	twoside,
         %invited,
	]{ieee}

\newcommand{\latexiie}{\LaTeX2{\Large$_\varepsilon$}}
\usepackage{url}
\usepackage{listings}
%Español
	\usepackage[utf8]{inputenc}
	\usepackage[spanish]{babel}
%loñapsE

%\usepackage{ieeetsp}	% if you want the "trans. sig. pro." style
%\usepackage{ieeetc}	% if you want the "trans. comp." style
%\usepackage{ieeeimtc}	% if you want the IMTC conference style

% Use the `endfloat' package to move figures and tables to the end
% of the paper. Useful for `submission' mode.
%\usepackage {endfloat}

% Use the `times' package to use Helvetica and Times-Roman fonts
% instead of the standard Computer Modern fonts. Useful for the 
% IEEE Computer Society transactions.
%\usepackage{times}
% (Note: If you have the commercial package `mathtime,' (from 
% y&y (http://www.yandy.com), it is much better, but the `times' 
% package works too). So, if you have it...
%\usepackage {mathtime}

% for any plug-in code... insert it here. For example, the CDC style...
%\usepackage{ieeecdc}

\begin{document}

%----------------------------------------------------------------------
% Title Information, Abstract and Keywords
%----------------------------------------------------------------------
\title[Entrega final práctica 1]{%
       Entrega final práctica 1 \\  Organización de Computadores}

% format author this way for journal articles.
% MAKE SURE THERE ARE NO SPACES BEFORE A \member OR \authorinfo
% COMMAND (this also means `don't break the line before these
% commands).
\author[]{Sebastián Arcila Valenzuela (\textit{sarcilav@eafit.edu.co}),
\and{} Ruben Dario Bueno Angel (\textit{rbuenoan@eafit.edu.co})
\and{}y Sergio Botero Uribe (\textit{sbotero2@eafit.edu.co}).
}

% format author this way for conference proceedings
%\author[PLETT AND KOLL\'{A}R]{%
        %Gregory L. Plett\member{Student Member},\authorinfo{%
        %Department of Electrical Engineering,\\ 
        %Stanford University, Stanford, CA 94305-9510.\\
        %Phone: $+$1\,650\,723-4769, email: glp@simoon.stanford.edu}%
%\and{}and%
%\and{}Istv\'{a}n Koll\'{a}r\member{Fellow}\authorinfo{%
        %Department of Measurement and Instrument Engineering,\\ 
        %Technical University of Budapest, 1521 Budapest, Hungary.\\
        %Phone: $+$\,36\,1\,463-1774, fax: +\,36\,1\,463-4112, 
        %email: kollar@mmt.bme.hu}
%}

\journal{ST0254-031 Organización de Computadores}
\titletext{, \today}
%\ieeecopyright{0018--9456/97\$10.00 \copyright\ 1997 IEEE}
%\lognumber{xxxxxxx}
%\pubitemident{S 0018--9456(97)09426--6}
%\loginfo{Manuscript received September 27, 1997.}
%\firstpage{0}

%\confplacedate{Ottawa, Canada, May 19--21, 1997}

\maketitle               

\begin{abstract} 
Entrega final de la práctica 1 de Organización de computadores, el informe contiene los detalles de la construcción del módulo así como las especificaciones del módulo, las dificultades encontradas y los comentarios por cada uno de los integrantes del equipo sobre la práctica.
\end{abstract}

\begin{keywords}
Práctica 1, organización de computadores, ensamblador, funciones trigonométricas, intel, nasm
\end{keywords}

\section{Introducción}
``In C++ and Java I experience a certain amount of angst when you ask how to do this and they say, "Well, you do it like this or you could do it like that." There are obviously too many features if you can do something that many ways-and they are more or less equivalent. I think there are smaller concepts that fit better in Inferno.''\cite{cita-thompson}

%----------------------------------------------------------------------
% SECTION I: Detalles de diseño
%----------------------------------------------------------------------
\section{Detalles de diseño de construcción}

Podemos decir que un algoritmo trivial para atacar las series de taylor seria el siguiente, tomemos como ejemplo de aquí en adelante la función seno:
\begin{lstlisting}[language=C++]
float sen(int x, int n)
//n numero de iteraciones 
{
 float xx = to_radianes(x);
 float ans = 0;
 for(int i = 0; i<n;++i)
 {
  ans += pow(-1,i)*pow(xx,2i+1)/(2i+1)!;
 }
 return ans;
}
\end{lstlisting}

Pero como podemos notar hay varias acotaciones que hacer a este código, primero no es necesario tener una función pow, ni una función factorial, puesto que cuando salgo de la iteración $m,(m>0)$ y paso a la iteración $m+1$, tenemos ya calculado hasta $xx^{2m+1}$ y el valor que necesitamos es $xx^{2m+3}$, entonces bastaría con solo multiplicar por $xx^{2}$  a $xx^{2m+1}$ para obtener el valor en $m+1$, casi de igual manera para el factorial cuando entramos en la iteración $m+1$ ya tenemos previamente el valor de $(2m+1)!$, y si nos fijamos el valor que necesitamos es $(2m+3)!$ que es igual a $(2m+1)!(2m+2)(2m+3)$, y para la situación del signo simplemente es en cada iteración hacer $signo = \neg signo$, y voila! no necesitamos funciones externas que desperdicien cálculos y tiempo; obteniendo algo similar a esto:

\begin{lstlisting}[language=C++]
float seno(int x, int n)
{
 float xx=to_radianes(x);
 float ans=0;
 float factorial=1;
 float acum_x2n=xx;
 int sign = -1;
 for(int i = 1; i<n;++i)
 {
   sign *= -1;
   ans += sign*acum_x2n/factorial;
   acum_x2n*=xx*xx;
   factorial*=(2*i)*(2*i+1);
 }
 return ans;
}

\end{lstlisting}
También vale la pena resaltar que los tipos de datos de factorial y de acum\_x2n es de tipo flotante, por que por obvias razones en números enteros da overflow para factorial y ademas los radianes ($xx$) que vamos a usar tienen precisión flotante.

Si lo notamos bien de la misma manera podemos deducir el siguiente algoritmo para coseno:
\begin{lstlisting}[language=C++]
float coseno(int x,int n)
{
 float xx=to_radianes(x);
 float ans=0;
 float factorial=1;
 float acum_x2n=1;
 int sign = -1;
 for(int i = 1; i<n;++i)
 {
  sign *= -1;
  ans += sign*acum_x2n/factorial;
  acum_x2n*=xx*xx;
  factorial*=(2*i-1)*(2*i);
 }
 return ans;
}

\end{lstlisting}

Ahora bien la operación de pasar de grados a radianes es lo más simple de este mundo, es simplemente multiplicar los grados por el factor de conversión que es ${{\pi}\over{180}} = 0.017453293$. Y estás serian todas la anotaciones respecto al diseño del algoritmo.






%----------------------------------------------------------------------
% SECTION II: Estructura de la práctica
%----------------------------------------------------------------------
\section{Especificaciones del módulo }
La práctica fue desarrollado para GNU/Linux y para Mac, la practica en GNU/Linux fue desarrollado en ubuntu 9.04 y en Mac OS X.5
\subsection{GNU/Linux}

Usamos nasm(NASM version 2.05.01 compiled on Nov  5 2008) para ensamblar nuestro código y generar el código objeto que se linkearia como una  Dynamically linked shared object libraries (.so)  con gcc (gcc versión 4.3.3 (Ubuntu 4.3.3-5ubuntu4) ).
Para ensamblar:
\begin{lstlisting}[language=bash]
nasm -f elf -o trigo.o trigo.s 
\end{lstlisting}
Para generar el .so:
\begin{lstlisting}[language=bash]
gcc -shared trigo.o -o libtrigo.so
\end{lstlisting}
Después de esto libtrigo.so debe ser llevado a /usr/lib y ya simplemente para compilar algo que haga uso de nuestra librería con gcc, solo necesitamos colocar -ltrigo en los flags de compilación y agregar trigo.h al directo de los include de la distribución en particular, en nuestro caso /usr/include, y agregar la cabecera \#$include<trigo.h>$ en el código en particular.

Las dependencias necesarias son: libglade2-dev libgtk2.0-bin.

Dependencias para construir desde los fuentes: libglade2-dev libgtk2.0-bin gcc make nasm

Para más detalles mirar la aplicación piloto que se puede encontrar en \url{http://code.assembla.com/trigo_assembla/subversion/nodes/branches/linux/app}, mirar el Makefile y trigonometria.c


% especificaciones_mac.tex
%
%  Created by Sergio on 2009-03-18.
%  Copyright (c) 2009 __MyCompanyName__. All rights reserved.
%
%\documentclass[]{article}
%\usepackage[utf8]{inputenc}
%\usepackage[english]{babel}

%\usepackage{listings}

%\begin{document}

Para usar el código de las funciones trigonométricas sobre Mac OS X Leopard, se debe primero generar un archivo objeto que tiene por formato Mach-O, un equivalente al ELF en Linux, en el cual aparecen todas librerías y programas. El objeto Mach-O puede tener código que puede ser `linkeado' dinámica y estáticamente.

Este objeto con los símbolos (en este caso las funciones) se crea con el ensamblador de NASM de la siguiente forma: 
\begin{lstlisting}[language=bash]
 $ nasm -f macho -o nasm.o nasm.s 
\end{lstlisting}
Después del cual obtenemos el objeto Mach-O que podremos usar en nuestro proyecto. El trabajo en OS X para probar la librería se dio en dos casos diferentes, el primero para probarla con una aplicación de línea de comando en Objective-C y la segunda segunda en una aplicación con interfaz gráfica usando Objective-C y Cocoa en donde para los dos cosos desde Objective-C las funciones de la librería se invocan de la misma forma que en C/C++, usando:\begin{lstlisting}[language=C]
extern float seno();
extern float coseno();
extern float tangente();
\end{lstlisting}
en los encabezados.
Para el caso de la aplicación en línea de comando la forma más efectiva para obtener el ejecutable es compilando el proyecto junto con el archivo objeto generado por NASM de la siguiente forma.
\begin{lstlisting}[language=bash]
 $ gcc Proyecto.m nasm.o -o ejecutable \ 
 	-framework Foundation
\end{lstlisting}
Para luego simplemente ejecutarlo.
Cuando se trabaja en Xcode, basta con importar el archivo objeto al proyecto y trabajar de la misma forma que se haría con la aplicación de línea de comando.
Existe la posibilidad de compilar y ejecutar el proyecto de Xcode mediante línea de comandos de la siguiente forma:
\begin{lstlisting}[language=bash]
 $ xcodebuild clean build install -activetarget \
 	-activeconfiguration
\end{lstlisting}
pero de todas formas se necesita importar el objeto generado por NASM desde Xcode.\\

{\bfseries NOTA:} El archivo `trigo.s' en Linux y `nasm\_dll.s' en OS X tienen una pequeña diferencia, ya que el ensamblador NASM de Linux permite que para las operaciones de flotantes se deje implícito el campo de parámetros para indicar que se está trabajando con las primeras direcciones de la pila, cosa que se debe dejar explícita en OS X, esa es la mayor diferencia en cuanto al código en ensamblador que se debe tener en cuenta para obtener la portabilidad de esta librería entre los dos sistemas operativos.

%\end{document}



%----------------------------------------------------------------------
% SECTION III: Dificultades
%----------------------------------------------------------------------
\section{Dificultades encontradas durante el diseño}

%
%  Created by Sergio on 2009-03-18.
%  Copyright (c) 2009 __MyCompanyName__. All rights reserved.
%
%\documentclass[]{article}
%\usepackage[utf8]{inputenc}
%\usepackage[spanish]{babel}
%\usepackage{fullpage}
%\begin{document}

La IDE FASM muy poco amigable y la documentación bastante incompleta, lo deja a uno a la deriva con frecuencia, debido a esto decidimos pasar a un IDE mejor documentado, el NASM, que está bastante maduro y tiene buena documentación en línea.
Documentación general de assembler para tópicos específicos dificil de "googliar", tal vez falta mas familiaridad con la escena ASM.
Una vez se pasó a NASM fué imposible crear la librería DLL para windows con el Linker.exe de Visual Basic, igual la funcionalidad en Windows era un extra pues el pensado era hacerla en Linux en donde quedó funcionando sin problemas.


\begin{flushright} 
	\itshape{-Ruben Dario Bueno Angel}
\end{flushright}
%\end{document}

%
%  Created by Sergio on 2009-03-18.
%  Copyright (c) 2009 __MyCompanyName__. All rights reserved.
%
%\documentclass[]{article}
%\usepackage[utf8]{inputenc}
%\usepackage[spanish]{babel}
%\usepackage{fullpage}
%\begin{document}

La principal dificultad que encontré para realizar la práctica, fue que en un principio cuando empezamos a trabajar con FASM, no tenía forma de ensamblar el código que escribiamos en mi equipo, por lo que los avances se daban muy lentamente al tener que probar el código siempre en otra parte. Después de la decisión de cambiar FASM, los inconvenientes estaban en pequeñas diferencias en sintaxis que variaban entre Linux y Mac, las cuales siempre que se quería ensamblar había que acomodar.


\begin{flushright} 
	\itshape{-Sergio Botero Uribe}
\end{flushright}
%\end{document}

El mayor dificultad fue escoger el ensamblador correcto.
Al principio intentamos mucho con fasm pero en vano, ponía mucho problema en el uso de las cabeceras y en la generación de código objeto sin mencionar que los errores que encontraba y devolvía eran casi incomprensibles.
Luego cuando tomamos la decisión de ensayar Gnu AS, funciona a las mil maravillas pero en sintaxis AT\&T, que a mi parecer es horrible, y aun asi usando el flag .intel\_syntax, no funcionaba nuestro programa escrito en asembler.
Pero en realidad todo empezó a funcionar con Netwide Assembler la sintaxis es intel por defecto y aparte de eso para ensamblar permite usar flags para distinguir el tipo de objeto que se debe hacer(-elf -macho). 
\begin{flushright} 
	\itshape{-Sebastián Arcila Valenzuela}
\end{flushright}

 
%----------------------------------------------------------------------
\section{Opiniones sobre la práctica}

%
%  Created by Sergio on 2009-03-18.
%  Copyright (c) 2009 __MyCompanyName__. All rights reserved.
%
%\documentclass[]{article}
%\usepackage[utf8]{inputenc}
%\usepackage[spanish]{babel}
%\usepackage{fullpage}
%\begin{document}

%
%
%  Created by Sergio on 2009-03-18.
%  Copyright (c) 2009 __MyCompanyName__. All rights reserved.
%
%\documentclass[]{article}
%\usepackage[utf8]{inputenc}
%\usepackage[spanish]{babel}
%\usepackage{fullpage}
%\begin{document}

La principal dificultad que encontré para realizar la práctica, fue que en un principio cuando empezamos a trabajar con FASM, no tenía forma de ensamblar el código que escribiamos en mi equipo, por lo que los avances se daban muy lentamente al tener que probar el código siempre en otra parte. Después de la decisión de cambiar FASM, los inconvenientes estaban en pequeñas diferencias en sintaxis que variaban entre Linux y Mac, las cuales siempre que se quería ensamblar había que acomodar.


\begin{flushright} 
	\itshape{-Sergio Botero Uribe}
\end{flushright}
%\end{document}


La práctica estuvo divertida e interesante aunque a ratos fué un poco frustrante, trabajar tan a bajo nivel presenta retos que normalmente se ignoran pero una vez superados dejan una sensación de satisfacción en vez de una sensación de rabia, lo cual sucede bastante seguido con lenguajes de alto nivel.

\begin{flushright} 
	\itshape{-Ruben Dario Bueno Angel}
\end{flushright}

%\end{document}


%
%  Created by Sergio on 2009-03-18.
%  Copyright (c) 2009 __MyCompanyName__. All rights reserved.
%
%\documentclass[]{article}
%\usepackage[utf8]{inputenc}
%\usepackage[spanish]{babel}
%\usepackage{fullpage}
%\begin{document}

%
%
%  Created by Sergio on 2009-03-18.
%  Copyright (c) 2009 __MyCompanyName__. All rights reserved.
%
%\documentclass[]{article}
%\usepackage[utf8]{inputenc}
%\usepackage[spanish]{babel}
%\usepackage{fullpage}
%\begin{document}

La principal dificultad que encontré para realizar la práctica, fue que en un principio cuando empezamos a trabajar con FASM, no tenía forma de ensamblar el código que escribiamos en mi equipo, por lo que los avances se daban muy lentamente al tener que probar el código siempre en otra parte. Después de la decisión de cambiar FASM, los inconvenientes estaban en pequeñas diferencias en sintaxis que variaban entre Linux y Mac, las cuales siempre que se quería ensamblar había que acomodar.


\begin{flushright} 
	\itshape{-Sergio Botero Uribe}
\end{flushright}
%\end{document}


Este tipo de programación en la que se le debe decir cada cosa al procesador, me parece que debió aparecer antes en la carrera, es mejor en cierta forma irse acomodando a los nuevos conceptos y a las nuevas formas de programación después de aprender lo que está más abajo.
La práctica no solo fue en lenguaje ensamblador, para poder hacer los aportes y completarla tuve que investigar cosas que a la final resultaron no tan claras como creía.
Un aspecto en el que quiero hacer mucho énfasis y que me pareció una idea que ojalá se repita para las próximas prácticas es la posibilidad y flexibilidad que se da para poder elegir la plataforma para hacer el trabajo, los incentivos son muy válidos pero más que eso es comprender que muchos estudiantes pueden sacar mucho más que el conocimiento y las habilidades que se espera obtengan del trabajo si tienen la posibilidad de trabajarla en donde les interesa hacer desarrollos, que para este caso particular, es desarrollo en Linux y por los lados en Mac. Puedo decir que esta práctica me ha incentivado a leer y aprender más cosas sobre la forma en que se trabaja más cerca del procesador en Mac.

\begin{flushright} 
	\itshape{-Sergio Botero Uribe}
\end{flushright}

%\end{document}


Este es el tipo de prácticas con las que uno aprende, pues nos toco que investigar y leer mucho por cuenta propia, por ejemplo las refencias para desarrolladores de intel, como generar librería en Unix, como linkear, como generar código objeto, como pegarle desde C a una librería hecha por nosotros. Además al programar en asembler me recuerda que lo bello es lo simple. En general puedo decir que me sentí programando como un hombre de verdad.

También quiero anotar que lo que mas me gusto es que no nos sesgaran a trabajar bajo una única plataforma, sino que nos insinuaron a probar en cosas diferentes mediante estímulos, aunque valederos, no enteramente necesarios para los que nos gusta aprender y hacer las cosas desde el ``lado de la luz'' (*nix).

\begin{flushright} 
	\itshape{-Sebastián Arcila Valenzuela}
\end{flushright}

%\end{document}


%----------------------------------------------------


\begin{thebibliography}{2}

\bibitem{Assembla -Free subversion Hosting}
\newblock \url{http://www.assembla.com/spaces/dashboard/index/trigo_assembla},
\newblock Sitio dondese encuentra todo el desarrollo y actividades de la práctica.
\newblock{Assembla -Free subversion Hosting- http://www.assembla.com}
\bibitem{cita-thompson}
\newblock \url{http://boole.computer.org/portal/site/computer/menuitem.eb7d70008ce52e4b0ef1bd108bcd45f3/index.jsp?&pName=computer_level1&path=computer/homepage/0599/thompson&file=thompson.xml&xsl=article.xsl&},Unix and Beyond: An Interview with Ken Thompson ,visitado Domingo 23 de agosto de 2009
\end{thebibliography}

%----------------------------------------------------------------------

\end{document}
